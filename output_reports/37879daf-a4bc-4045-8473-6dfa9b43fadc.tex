\documentclass{article}
\usepackage[utf8]{inputenc}
\usepackage{booktabs}

\title{Reporte Ejecutivo de Resultados Financieros}
\author{Nombre de la Empresa}
\date{\today}

\begin{document}

\maketitle

\begin{abstract}
Este reporte presenta un análisis de los resultados financieros del mes, destacando las ventas, costos y utilidad generada. Se incluye una tabla con los resultados y un breve análisis de la situación financiera.
\end{abstract}

\section{Resultados Financieros}

A continuación se presentan los resultados financieros del mes:

\begin{table}[h]
    \centering
    \begin{tabular}{@{}lc@{}}
        \toprule
        Concepto & Monto (\$) \\ \midrule
        Ventas   & 20,000     \\
        Costos   & 12,000     \\
        Utilidad & 8,000      \\ \bottomrule
    \end{tabular}
    \caption{Resultados Financieros del Mes}
    \label{tab:resultados}
\end{table}

\section{Análisis Financiero}

La utilidad generada este mes es de \$8,000, lo que representa un aumento del 15\% respecto al mes anterior. Este incremento en la utilidad se debe a un control más eficiente de los costos y un aumento en las ventas, que alcanzaron los \$20,000. Los costos se mantuvieron en \$12,000, lo que permitió mejorar el margen de utilidad.

\section{Conclusión}

Los resultados financieros muestran una tendencia positiva en la utilidad, lo que indica una buena gestión de los recursos y un crecimiento en las ventas. Se recomienda continuar con las estrategias implementadas para mantener este crecimiento y seguir optimizando los costos.

\end{document}