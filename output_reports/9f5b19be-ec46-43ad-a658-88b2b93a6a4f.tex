\documentclass{article}
\usepackage[utf8]{inputenc}
\usepackage{booktabs}

\title{Reporte Ejecutivo de Resultados Financieros}
\author{Algo Nuevo}
\date{\today}

\begin{document}

\maketitle

\begin{abstract}
Este reporte presenta un análisis financiero de la empresa "Algo Nuevo" correspondiente al último mes. Se detallan las ganancias, ventas, costos y utilidad, así como un comentario sobre el desempeño financiero en comparación con el mes anterior.
\end{abstract}

\section{Resultados Financieros}

A continuación se presentan los resultados financieros de la empresa:

\begin{table}[h]
    \centering
    \begin{tabular}{@{}ll@{}}
        \toprule
        \textbf{Concepto} & \textbf{Valor} \\ \midrule
        Ganancia          & \$200          \\
        Ventas           & \$20,000       \\
        Costos           & \$12,000       \\
        Utilidad         & \$8,000        \\ \bottomrule
    \end{tabular}
    \caption{Resultados Financieros de la Empresa "Algo Nuevo"}
    \label{tab:resultados}
\end{table}

\section{Análisis Financiero}

La utilidad de la empresa ha mostrado un crecimiento significativo, aumentando un 15\% respecto al mes anterior. Este incremento es un indicador positivo del desempeño de la empresa, sugiriendo una gestión eficiente de los costos y un aumento en las ventas. 

La relación entre las ventas y los costos se mantiene favorable, lo que permite a la empresa generar una utilidad sólida. La estrategia de ventas implementada parece estar dando resultados, y es recomendable continuar con las acciones que han llevado a este crecimiento.

\section{Conclusión}

En resumen, la empresa "Algo Nuevo" ha tenido un mes exitoso, con un aumento en la utilidad y un control adecuado de los costos. Se sugiere seguir monitoreando estos indicadores y ajustar las estrategias según sea necesario para mantener el crecimiento sostenido.

\end{document}