\documentclass{article}
\usepackage[utf8]{inputenc}
\usepackage{booktabs}

\title{Reporte Ejecutivo de Resultados Financieros}
\author{Mi gato}
\date{\today}

\begin{document}

\maketitle

\section*{Resumen}
Este reporte presenta un análisis financiero de la empresa "Mi gato" correspondiente al último mes. Se detallan las ventas, costos y utilidad, así como un comentario sobre el desempeño financiero en comparación con el mes anterior.

\section*{Resultados Financieros}
A continuación se presenta una tabla con los resultados financieros de la empresa:

\begin{table}[h]
    \centering
    \begin{tabular}{@{}lc@{}}
        \toprule
        \textbf{Concepto} & \textbf{Monto (en \$)} \\ \midrule
        Ventas            & 20,000                \\
        Costos            & 12,000                \\
        Utilidad          & 8,000                 \\ \bottomrule
    \end{tabular}
    \caption{Resultados Financieros de la Empresa "Mi gato"}
    \label{tab:resultados}
\end{table}

\section*{Análisis Financiero}
La utilidad de la empresa "Mi gato" ha alcanzado un total de \\$8,000, lo que representa un aumento del 15\\% en comparación con el mes anterior. Este crecimiento en la utilidad es un indicador positivo del desempeño de la empresa, sugiriendo una gestión eficiente de los costos y un aumento en las ventas.

\end{document}