\documentclass{article}
\usepackage[utf8]{inputenc}
\usepackage{booktabs}

\title{Reporte Ejecutivo de Resultados Financieros}
\author{Mi gato feliz}
\date{\today}

\begin{document}

\maketitle

\section*{Resumen}
Este reporte presenta un análisis financiero de la empresa "Mi gato feliz" correspondiente al último mes. Se detallan las ventas, costos y utilidad, así como un breve comentario sobre el desempeño financiero en comparación con el mes anterior.

\section*{Resultados Financieros}
A continuación se presenta una tabla con los resultados financieros de la empresa:

\begin{table}[h]
    \centering
    \begin{tabular}{@{}lc@{}}
        \toprule
        Concepto & Monto (\$) \\ \midrule
        Ventas   & 20,000     \\
        Costos   & 12,000     \\
        Utilidad & 8,000      \\ \bottomrule
    \end{tabular}
    \caption{Resultados Financieros de Mi gato feliz}
    \label{tab:resultados}
\end{table}

\section*{Análisis Financiero}
La empresa "Mi gato feliz" ha reportado ventas de \$20,000 y costos de \$12,000, lo que resulta en una utilidad de \$8,000. Este resultado representa un aumento del 15\% en la utilidad respecto al mes anterior, lo que indica una mejora en la eficiencia operativa y en la gestión de costos. 

Este crecimiento en la utilidad es un indicador positivo que sugiere que las estrategias implementadas están dando resultados favorables. Se recomienda continuar monitoreando estos indicadores y explorar oportunidades para incrementar aún más las ventas y optimizar los costos.

\section*{Conclusión}
El desempeño financiero de "Mi gato feliz" ha sido satisfactorio, con un aumento notable en la utilidad. Es fundamental seguir evaluando las estrategias actuales y considerar nuevas iniciativas que puedan contribuir al crecimiento sostenido de la empresa.

\end{document}